% !TEX encoding = UTF-8 Unicode
\documentclass{article}
\newcommand{\hwnumber}{5*}

\newcommand{\norm}[1]{\| #1 \|}
\newcommand{\abs}[1]{| #1 |}


\usepackage{fullpage,amsthm,amsmath,amssymb}
\usepackage{algorithm,algorithmic}
\usepackage{mathtools}
\usepackage{bbm,bm}
\usepackage{enumerate}
\usepackage{cancel}
\usepackage{xspace}
\usepackage[textsize=tiny,
%disable
]{todonotes}
\newcommand{\todot}[1]{\todo[color=blue!20!white]{T: #1}}
\newcommand{\todoc}[1]{\todo[color=orange!20!white]{Cs: #1}}
\usepackage[colorlinks,citecolor=blue,urlcolor=blue,linkcolor=black]{hyperref}
\newcommand\numberthis{\addtocounter{equation}{1}\tag{\theequation}}
\usepackage[capitalize]{cleveref}


\usepackage{comment}

\newcommand{\R}{\mathbb{R}}
\newcommand{\cZ}{\mathcal{Z}}
\newcommand{\cX}{\mathcal{X}}
\DeclareMathOperator*{\argmin}{argmin}
\DeclareMathOperator{\sign}{sign}
\DeclareMathOperator*{\Exp}{\mathbb{E}}
\newcommand{\E}{\mathbb{E}}
\DeclareMathOperator*{\1}{\mathbbm{1}}
\newcommand{\set}[1]{\left\{#1\right\}}
\newcommand{\bbP}{\mathbb P}
\newcommand{\V}{\mathbb V}
\renewcommand{\P}[1]{P\left\{ #1 \right\}}
\newcommand{\Prob}[1]{\mathbb{P}( #1 )}
\newcommand{\real}{\mathbb{R}}
\renewcommand{\b}[1]{\mathbf{#1}}
\newcommand{\EE}[1]{\E[#1]}
\newcommand{\bfone}{\1}
\newcommand{\one}[1]{\mathbb{I}\{#1\}}
\newcommand{\NN}{\mathbb{N}}
\newcommand{\cF}{\mathcal{F}}
\newcommand{\0}{\mathbf{0}}
\usepackage{xifthen}

% total number of points that can be collected
\newcounter{DocPoints} % counter reset to zero upon creation

% counting points per question (not user facing)
\newcounter{QuestionPoints} % counter reset to zero upon creation

% Points for a subquestion of a question; 
% Adds the points to the total for the question and the document.
\newcommand{\points}[1]{%
	\par\mbox{}\par\noindent\hfill {\bf #1 points}%
	\addtocounter{DocPoints}{#1}
	\addtocounter{QuestionPoints}{#1}
}
% Points for a question; call with no params if the question
% had subquestions. In this case it prints the total for the question (see \points).
% Otherwise call with the points that the question is worth.
% In this case, the total is added to the document total.
% It is a semantic error to call this with a non-empty parameter
% when a question had subquestions with individual scores.
\newcommand{\tpoints}[1]{        %
	\ifthenelse{\isempty{#1}}%
	{%
	}%
	{%
		\addtocounter{DocPoints}{#1}
		\addtocounter{QuestionPoints}{#1}
	}													 %
	\par\mbox{}\par\noindent\hfill {Total: \bf \arabic{QuestionPoints}\xspace points}\par\mbox{}\par\hrule\hrule
	\setcounter{QuestionPoints}{0}
}
\newcommand{\tpoint}[1]{
	\tpoints{#1}
}

\theoremstyle{definition}
\newtheorem{question}{Question}
\newtheorem{definition}{Definition}
\newtheorem{assumption}{Assumption}
\newtheorem*{assumption*}{Assumption}

\theoremstyle{remark}
\newtheorem{remark}{Remark}
\newtheorem*{remark*}{Remark}
\newtheorem{solution}{Solution}
\newtheorem*{solution*}{Solution}

\theoremstyle{theorem}
\newtheorem{theorem}{Theorem}
\newtheorem{lemma}{Lemma}
\newtheorem{proposition}{Proposition}

\excludecomment{solution}
%\excludecomment{solution*}

\newcommand{\hint}{\noindent \textbf{Hint}:\xspace}


\usepackage{hyperref}

\newcommand{\epssub}{\delta}
\newcommand{\cH}{\mathcal{H}}
\newcommand{\sA}{\mathcal{A}}
\newcommand{\cS}{\mathcal{S}}
\newcommand{\cA}{\mathcal{A}}
\newcommand{\cB}{\mathcal{B}}


\begin{document}

\begin{center}
{\Large \textbf{CMPUT 653: Theoretical Foundations of Reinforcement Learning, Winter 2021\\ Homework \#\hwnumber}}
\end{center}

\section*{Instructions}
\textbf{Submissions}
You need to submit a zip file, named {\tt p\hwnumber\_<name>.zip} 
or {\tt p\hwnumber\_<name>.pdf} 
where {\tt <name>} is your name.
The zip file should include a report in PDF, typed up (we strongly encourage to use pdf\LaTeX) and the code that we asked for. Write your name on your solution.
I provide a template that you are encouraged to use.
You have to submit the zip file on the eclass website of the course.

\textbf{Collaboration and sources}
Work on your own. You can consult the problems with your classmates, use books
or web, papers, etc.
Also, the write-up must be your own and you must acknowledge all the
sources (names of people you worked with, books, webpages etc., including class notes.) 
Failure to do so will be considered cheating.  
Identical or similar write-ups will be considered cheating as well.
Students are expected to understand and explain all the steps of their proofs.

\textbf{Scheduling}
Start early: It takes time to solve the problems, as well as to write down the solutions. Most problems should have a short solution (and you can refer to results we have learned about to shorten your solution). Don't repeat calculations that we did in the class unnecessarily.

\vspace{0.3cm}

\textbf{Deadline:} April 6 at 11:55 pm

\newcommand{\cM}{\mathcal{M}}
\newcommand{\nS}{\mathrm{S}}
\newcommand{\nA}{\mathrm{A}}
\newcommand{\PP}{\mathbb{P}}
\newcommand{\RR}{\mathbb{R}}
\newcommand{\ip}[1]{\langle #1 \rangle}
\newcommand{\N}{\mathbb{N}}
\newcommand{\cG}{\mathcal{G}}
\newcommand{\cP}{\mathcal{P}}

\section*{Sufficient condition for policy gradients to exist}

In
\href{https://rltheory.github.io/lecture-notes/planning-in-mdps/lec16/}
{Lecture 16} 
it is stated that a sufficient condition for the differentiability of $x \mapsto J(\pi_x)$ at $x=\theta_0$ 
in a finite MDP 
is that for 
$x\mapsto \pi_x(a|s)$ is continuously differentiable at $x=\theta_0$ for any $(s,a)$ pair.
Our purpose here is to show that this is correct.

We start with a
more general sufficient condition that allows for infinite state and action spaces
and would be applicable, for example, for LQR problems with the discounted total reward criterion.
The next question will be concerned with the (simpler) finite case.

\begin{question}
Let $M = (\cS,\cA,P,r)$ be an MDP so that for any memoryless policy $\pi$, $q^\pi$ exists.
For $x\in \mathbb{R}^d$ let $\pi_x$ be a memoryless policy.
Let $\mu \in \cM_1(\cS)$ and for a memoryless policy let $J(\pi) = \mu v^\pi$.
For $\mu'\in \cM_1(\cS)$ and bounded $q: \cS \times \cA \to \R$, 
let $F(\mu',x,q) = \mu' M_{\pi_x} q$.
Assume that for any fixed $\mu',q$, $F(\mu',\cdot,q)$ 
 has continuous partial derivatives at $x=\theta_0\in \mathbb{R}^d$ 
and that
\begin{align*}
(\mu',q) \mapsto \sum_{i=1}^d \frac{\partial}{\partial x_i} F(\mu',x,q)|_{x=\theta_0} e_i
\end{align*}
is continuous in a neighborhood of $( (1-\gamma)\tilde \nu_\mu^{\pi_{\theta_0}}, q^{\pi_{\theta_0}} )$
(here $e_i$ is the $i$th vector in the standard Euclidean basis).
 Then the following hold:
\begin{enumerate}
\item $x\mapsto J(\pi_x)$ is differentiable;
\item the conditions of the policy gradient theorem are met.
\end{enumerate}
\tpoints{50}
\end{question}
\begin{solution*}
We first show the second part.
In particular, we show that 
{\em (a)} $\theta \mapsto f_{\pi_\theta}'(\theta_0)$ exists and is continuous in a neighborhood of $\theta_0$ and {\em (b)} $g_{\pi_{\theta_0}}'(\theta_0)$ exists,
 where
\begin{align*}
f_\pi(x) &= \tilde\nu_\mu^\pi M_{\pi_x} q^\pi\,,\\
g_\pi(x) &=\tilde \nu_\mu^{\pi_x} v^\pi\,.
\end{align*}
Then, the policy gradient theorem itself shows that Part 1 is true.

Let us start with Claim (a).
Fix a memoryless policy $\pi$. 
Since $q^\pi$ is bounded and $(1-\gamma)\nu_\mu^\pi\in \cM_1(\cS)$, 
by our assumption
the partial derivatives of $f_\pi(x)$ exist and are continuous.
Hence, $f_\pi$ is differentiable and

\begin{align*}
\frac{d}{dx}
\tilde\nu_\mu^\pi M_{\pi_x} q^\pi
= 
\sum_{i=1}^d
\frac{\partial}{\partial x_i} \tilde\nu_\mu^\pi M_{\pi_x} q^\pi e_i\,.
\end{align*}

Since by assumption the right-hand side is continuous in a neighborhood 

For $\theta$ in a neighborhood of $\theta_0$, we need that 
$f_{\pi_\theta}$ is differentiable at $x=\theta_0$.
Fix $\theta$.



Claim (a) follows since by our assumption $x\mapsto \tilde \nu_\mu^{\pi_{\theta_0}} M_{\pi_x} q^{\pi_{\theta_0}}$ has continuous partial derivatives \todoc{really?} in a neighborhood of $\theta_0$,
hence it is differentiable in a neighborhood of $\theta_0$.


For brevity let $v = v^{\pi_{\theta_0}}$.
We have
\begin{align*}
J(\pi_x) = 
%\sum_{t\ge 0} \gamma^t \mu (P_{\pi_x))^t  M_\pi r.
\end{align*}
\qed\par\smallskip\hrule
\end{solution*}


%\begin{question}
%\end{question}
%\begin{solution*}
%\qed\par\smallskip\hrule
%\end{solution*}


\bigskip
\bigskip

\noindent
\textbf{
Total for all questions: \arabic{DocPoints}}.
Of this, $100$ are bonus marks (i.e., $100$ marks worth $100\%$ on this problem set).

\end{document}




