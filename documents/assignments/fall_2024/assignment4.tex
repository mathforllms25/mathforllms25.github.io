\documentclass{article}
\newcommand{\hwnumber}{3--4}

\newcommand{\norm}[1]{\| #1 \|}
\newcommand{\abs}[1]{| #1 |}
\newcommand{\eps}{\varepsilon}
\renewcommand{\epsilon}{\varepsilon}

\usepackage{fullpage,amsthm,amsmath,amssymb}
\usepackage{algorithm,algorithmic}
\usepackage{mathtools}
\usepackage{bbm,bm}
\usepackage{enumerate}
\usepackage{xspace}
\usepackage[textsize=tiny,
]{todonotes}
\newcommand{\todot}[1]{\todo[color=blue!20!white]{T: #1}}
\newcommand{\todoc}[1]{\todo[color=orange!20!white]{Cs: #1}}
\usepackage[colorlinks,citecolor=blue,urlcolor=blue,linkcolor=black]{hyperref}
\newcommand\numberthis{\addtocounter{equation}{1}\tag{\theequation}}

\newtheorem{theorem}{Theorem}
\newtheorem{corollary}{Corollary}

\newcommand{\R}{\mathbb{R}}
\DeclareMathOperator*{\argmin}{argmin}
\DeclareMathOperator{\sign}{sign}
\DeclareMathOperator{\Var}{Var}
\DeclareMathOperator{\Lap}{Lap}
\DeclareMathOperator*{\Exp}{\mathbf{E}}
\DeclareMathOperator*{\1}{\mathbbm{1}}
\newcommand{\set}[1]{\left\{#1\right\}}
\newcommand{\E}{\mathbb E}
\newcommand{\V}{\mathbb V}
\renewcommand{\P}[1]{P\left\{ #1 \right\}}
\newcommand{\Prob}[1]{\mathbb{P}( #1 )}
\newcommand{\Probgr}[1]{\mathbb{P}\left( #1 \right)}
\newcommand{\real}{\mathbb{R}}
\renewcommand{\b}[1]{\mathbf{#1}}
\newcommand{\EE}[1]{\E[#1]}
\newcommand{\bfone}{\1}
\newcommand{\NN}{\mathbb{N}}
\newcommand{\cF}{\mathcal{F}}
\usepackage[capitalize]{cleveref}
\usepackage{xifthen}

\newcounter{DocPoints} 
\newcounter{QuestionPoints} 
\newcommand{\points}[1]{	\par\mbox{}\par\noindent\hfill {\bf #1 points}	\addtocounter{DocPoints}{#1}
	\addtocounter{QuestionPoints}{#1}
}
\newcommand{\tpoints}[1]{        	\ifthenelse{\isempty{#1}}	{	}	{		\addtocounter{DocPoints}{#1}
		\addtocounter{QuestionPoints}{#1}
	}													 	\par\mbox{}\par\noindent\hfill {Total: \bf \arabic{QuestionPoints}\xspace points}\par\mbox{}\par\hrule\hrule
	\setcounter{QuestionPoints}{0}
}
\newcommand{\tpoint}[1]{
	\tpoints{#1}
}

\theoremstyle{definition}
\newtheorem{question}{Question}

\theoremstyle{remark}
\newtheorem{remark}{Remark}
\newtheorem*{remark*}{Remark}
\newtheorem{solution}{Solution}
\newtheorem*{solution*}{Solution}

\newcommand{\hint}{\noindent \textbf{Hint}:\xspace}

\usepackage{hyperref}

\newcommand{\epssub}{\delta}
\newcommand{\cH}{\mathcal{H}}
\newcommand{\sA}{\mathcal{A}}
\newcommand{\cS}{\mathcal{S}}
\newcommand{\cA}{\mathcal{A}}
\newcommand{\cB}{\mathcal{B}}


\begin{document}

\begin{center}
{\Large \textbf{CMPUT 654: Theoretical Foundations of Machine Learning, Fall 2023\\ Homework \#\hwnumber}}
\end{center}

\section*{Instructions}
\textbf{Submissions}
You need to submit a single PDF file, named {\tt p0\hwnumber\_<name>.pdf} where {\tt <name>} is your name.
The PDF file should include your typed up solutions (we strongly encourage to use pdf\LaTeX). 
Write your name in the title of your PDF file.
We provide a \LaTeX template that you are encouraged to use.
To submit your PDF file you should send the PDF file via private message to Csaba on Slack before the deadline.


\textbf{Collaboration and sources}
Work on your own. You can consult the problems with your classmates, use books
or web, papers, etc.
Also, the write-up must be your own and you must acknowledge all the
sources (names of people you worked with, books, webpages etc., including class notes.) 
Failure to do so will be considered cheating.  
Identical or similar write-ups will be considered cheating as well.
Students are expected to understand and explain all the steps of their proofs.

\textbf{Scheduling}
Start early: It takes time to solve the problems, as well as to write down the solutions. Most problems should have a short solution (and you can refer to results we have learned about to shorten your solution). Don't repeat calculations that we did in the class unnecessarily.

\textbf{Other}
Some problems get zero points. These are practice problems that will not be marked.

\vspace{0.3cm}

\textbf{Deadline:} December 10 at 11:55 pm

\newcommand{\cM}{\mathcal{M}}
\newcommand{\nS}{\mathrm{S}}
\newcommand{\nA}{\mathrm{A}}
\newcommand{\PP}{\mathbb{P}}
\newcommand{\RR}{\mathbb{R}}
\newcommand{\cX}{\mathcal{X}}
\newcommand{\cE}{\mathcal{E}}
\newcommand{\cY}{\mathcal{Y}}
\newcommand{\cZ}{\mathcal{Z}}
\newcommand{\cG}{\mathcal{G}}
\newcommand{\cD}{\mathcal{D}}
\newcommand{\cN}{\mathcal{N}}
\newcommand{\ip}[1]{\langle #1 \rangle}
\newcommand{\one}[1]{\mathbb{I}\{#1\}}
\newcommand{\KL}{\mathrm{KL}}



\section*{Problems}

\begin{question}
Solve Exercise 5.1.
\tpoints{10}
\end{question}




\begin{question}
Solve Exercise 5.3.
\tpoints{10}
\end{question}




\begin{question}
Show that the claim made in Example 6.27 is true.
Show your work.
\tpoints{20}
\end{question}




\begin{question}
Solve Exercise 8.6, i.e., prove Proposition 8.12.
\tpoints{40}
\end{question}




\begin{question}
Show that the result claimed in Example 9.23 indeed holds. As usual, show your work.
\tpoints{20}
\end{question}




\begin{question}
Solve Exercise 9.5 from the book.
\tpoints{20}
\end{question}





\bigskip
\bigskip

\noindent
\textbf{
Total for all questions: \arabic{DocPoints}}.
Of this, \textcolor{red}{20} are bonus marks. 
Your assignment will be marked out of \textcolor{red}{100}.


\end{document}





